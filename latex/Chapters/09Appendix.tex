% !TeX root = ..\main.tex
\section*{Github repository}
\addcontentsline{toc}{chapter}{\protect\numberline{}A - Github repository} 

All code files used in this document are included in the Github repository linked below.

\begin{itemize}
    \item \url{https://github.com/eirikfagerbakke/specialization_project}
\end{itemize}

\section*{Notation summary}
\addcontentsline{toc}{chapter}{\protect\numberline{}B - Notation summary}
\begin{table}[h!]
\centering
\begin{tabular}{c|c}
\textbf{Symbol} & \textbf{Description} \\
\hline
\(\mathcal{S}\) & Operator mapping from input to output function space \\
\(\mathcal{A}\) & Input function space (Banach space) \\
\(\mathcal{U}\) & Output function space (Banach space) \\
\(\mathcal{X}\) & Bounded spatial domain \(\subset \mathbb{R}^d\) \\
\(\mathcal{Y}\) & Spatial-temporal domain \(\mathcal{T} \times \mathcal{X} \subset \mathbb{R}^{d+1}\) \\
\(a\) & Input function (initial condition of the PDE) \\
\(u\) & Output function, \(u = \mathcal{S}[a]\) \\
\(y\) & Query point, \(y = (x,t) \in \mathbb{R}^{d+1}\) \\
\(\{a^{(i)}, u^{(i)}\}_{i=1}^N\) & Observations of input-output pairs \\
\(\mu\) & Probability measure supported on \(\mathcal{A}\) \\
\(\mathcal{S}_\theta\) & Neural network approximation of the operator \(\mathcal{S}\) \\
\(\theta\) & Parameters of the neural network \(\in \mathbb{R}^p\) \\
\(\lambda\) & Self-adaptive weights
\end{tabular}
\caption{Summary of notation used in the project.}
\label{tab:notation}
\end{table}


%%%%%%%%%%%%%%%%%%%%%%%%%%%%%%%%%%%%%%%%%%%%%%%%%%%%%%%%